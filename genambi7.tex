% Last edit MW, 14/12/2020.

\documentclass[9pt,lineno]{elife}
% Use the onehalfspacing option for 1.5 line spacing Use the doublespacing
% option for 2.0 line spacing Please note that these options may affect
% formatting.  Additionally, the use of the \newcommand function should be
% limited.

\usepackage{lipsum} % Required to insert dummy text
\usepackage[version=4]{mhchem} \usepackage{siunitx} \DeclareSIUnit\Molar{M}

% amsmath and amssymb packages, useful for mathematical formulas and symbols

\usepackage{graphicx}
\usepackage{color}
\usepackage{subfigure}
\usepackage{amsmath,amssymb}
\usepackage{hyperref}

\newcommand{\MW}[1]{{\color{magenta}{#1}}}
\newcommand{\DY}[1]{{\color{blue}{#1}}}
\newcommand{\GH}[1]{{\color{green}{#1}}}

%%%%%%%%%%%%%%%%%%%%%%%%%%%%%%%%%%%%%%%%%%%%%%%%%%%%%%%%%%%% %% ARTICLE SETUP
%%%%%%%%%%%%%%%%%%%%%%%%%%%%%%%%%%%%%%%%%%%%%%%%%%%%%%%%%%%%

\title{Polymorphism of Genetic Ambigrams}

\author[1]{Gytis Dudas}
\author[2]{Greg Huber}
\author[2,3,*]{Michael Wilkinson}
\author[2]{David  Yllanes}


\affil[1]{Gothenburg Global Biodiversity Centre, Carl Skottsbergs gata 22B, 413 19, Gothenburg, Sweden}
\affil[2]{Chan Zuckerberg Biohub, 499 Illinois Street, San Francisco, CA 94158, USA}
\affil[3]{School of Mathematics and Statistics, The Open University, Walton Hall, 
Milton Keynes, MK7 6AA, UK}

\corr{gytisdudas@gmail.com}{GD}
\corr{greg.huber@czbiohub.org}{GH}
\corr{michael.wilkinson@czbiohub.org}{MW} 
\corr{david.yllanes@czbiohub.org}{DY}


\begin{document} \maketitle

\begin{abstract}

Double-synonyms in the genetic code can be used as a tool to
test competing hypotheses regarding ambigrammatic 
narnavirus genomes. Applying the analysis to recent observations of polymorphs of 
an ambigrammatic virus indicates that most of the open reading frame on the 
complementary strand does \emph{not} code for a functional protein. This ambigrammatic 
gene was found to be associated with an apparently symbiotic companion RNA 
molecule, also ambigrammatic, termed \lq Robin'. Our analysis of the polymorphism of Robin suggests 
that it does not code for a protein. We make a hypothesis about its role.  
\end{abstract}

\section{Introduction}
\label{sec: 1}
Of all the various types of viruses catalogued, narnaviruses rank among the simplest
and most surprising~\cite{Cob+16}.  Narnaviruses (a contraction of \lq naked RNA virus')
are examples of a minimal blueprint for a virus: no capsid, no envelope, no apparent
assembly of any kind. The known narnaviruses appear to be single genes, which code for an
RNA-dependent RNA polymerase
(abbreviated as RdRp) \cite{Hillman2013}. Some narnaviruses 
are found to have a genome for which there is an open reading frame (that is, a reading frame without 
stop codons) on the strand which is complementary to that which codes the gene for the RdRp. 
This reverse open reading frame has codon boundaries \MW{aligned} with the forward reading 
frame. Because the genome can be read in either direction, we say that these narnaviruses 
are \emph{ambigrammatic}. The significance of an ambigrammatic genome is an open problem. 
In this paper we discuss how observations on the polymorphism of the genetic code can distinguish 
between competing hypotheses on the function and nature of ambigrammatic viral genomes.

Our discussion is based upon two rules about the genetic code and its relation to ambigrammatic 
sequences. Both of these \emph{ambigram rules} are concerned with the availability of synonyms within 
the genetic code, which allow coding of the same amino acid with a different codon. 
The first rule states that for any sequence of amino acids coded by the forward strand, 
it is possible to use synonym substitutions which result from single-base mutations in order to remove 
all stop codons on the complementary strand (this result was discussed already in \cite{DeR+19}). 
The second ambigram rule, described below, states that the genetic code contains double 
synonyms that allow polymorphisms, accessible by single-base mutations, even when the 
amino acids coded by both the forward and the complementary strands are fixed. 

The first of these rules addresses the \lq how' of  ambigrammatic genomes, by showing that 
stop codons on the complementary strand can be removed by single-point mutations, without 
altering the protein (in narnaviruses, the RdRp) coded on the forward gene. Here we argue that the 
second rule can help to resolve the \lq why' of ambigrammatic genomes: 
the origin of ambigrammaticity itself. There are two 
distinct reasons why there might be an evolutionary advantage for a virus to evolve an 
ambigrammatic sequence. The first possibility is that the complementary strand might code 
for a functionally significant protein, for example, one that might poison the defence mechanisms 
of the host cell. The second possibility is that the lack of stop codons on the complementary strand 
is significant, even if the amino acid sequence that is coded is irrelevant. In particular, the lack of stop codons 
may promote the association between ribosomes and the complementary strand viral RNA (produced as part of its 
replication cycle). It is possible that 
a \lq polysome' formed by a covering of ribosomes helps to shield 
the virus from detection by cellular defence mechanisms. The second ambigram rule combined
with data on the polymorphism of the 
virus genome can help distinguish whether the complementary strand codes for a functional protein. 
We shall argue that the preliminary evidence is in favour of this second hypothesis, namely that 
most of the open reading frame on the complementary strand does not code for a functional
protein.

After describing the genetic ambigram rules, we discuss how the existence of double synonyms 
can be used to assess whether the open reading frame on the complementary chain codes for 
functional protein. It is well known that, because RdRp is a highly-conserved gene, synonymous 
mutations occur more frequently than non-synonymous ones. Some of these synonymous 
mutations have the potential to be synonymous in the complementary strand. If the complementary 
strand also codes for a functional protein, we expect that doubly synonymous mutations will be 
favoured. In fact, there would be mutational \lq hotspots' corresponding to the potential 
doubly-synonymous loci. We introduce two tests for whether the complementary strand 
is coding, based respectively on looking for mutational \lq hotspots', and upon the mutational
frequencies at loci which have double-synonyms. We used these tests to analyse sequences for 
$43$ polymorphic variants of an ambigrammatic narnavirus, using data reported in \cite{Bat+20}. 
We find that neither of our tests supports the hypothesis that the translated sequence of the 
complementary strand sequence 
is under selective pressure. We also apply these tests to a second ambigrammatic RNA sequence, 
termed the \emph{Robin} sequence, which is closely associated with ambigrammatic narnavirus infection 
in mosquitos. We find that neither the of the two complementary open reading frames 
of Robin appears to be under selective pressure for its amino acid sequence. 
In the concluding section, we consider the interpretation of these observations, and discuss whether there 
may be implications for other viral families.

\MW{There are many examples of overlapping genes with staggered reading frames, and recent 
work by Nelson, Ardern and Wei \cite{Nel+20} discusses how these can be identified. 
Our investigations indicate that the ambigrammatic genes discussed in this work are a 
different phenomenon, because they are non-coding. Our approach to analysing the ambigrammatic 
sequences is quite distinct from the rather complex machinery proposed in \cite{Nel+20}, because 
it emphasises the role of double synonyms as an unambiguous discriminant of the role of the ambigrammatic 
sequences.}

\section{Ambigram rules and their significance}
\label{sec: 2}

We start by describing the two genetic ambigram rules.

\subsection{First rule: all complementary-strand stops are removable} 
\label{sec: 2.1}

Consider the reading frame on the complementary strand that has its 
codons aligned with those on the forward strand. Every codon on the forward 
strand corresponds to a complementary-strand codon read in the reverse 
direction. The rule states that any stop codon on the complementary strand 
can be removed by a single-point mutation which leaves the amino acid specified by the 
forward-read codon unchanged. 

This result is demonstrated by the following argument, as discussed in \cite{DeR+19}. Reversing the read direction 
and taking the pairing complement, the stop codons UAA, UAG, UGA become, respectively, 
UUA, CUA, UCA, for which the amino acids are Leu, Leu, Ser. It is only instances of leucine and 
serine in the forward sequence that can result in stop codons in the reverse read. 
The synonyms of Leu are  CU*,  UUA, UUG (where * means any base).
The synonyms of Ser are UC*, AGU, AGC. The undesirable Leu codon UUA can be transformed 
to UUG by a single substitution. Similarly, the Leu codon CUA can be transformed to 
CUU, CUG or CUC by single substitutions. And the Ser codon UCA is transformed to UCU, UCG or UCC
by single substitutions.

Furthermore, it is found that complementary-strand stops cannot always be removed by synonym substitutions 
in the other two read frames for the complementary strand (this requires a longer argument, also
given in \cite{DeR+19}). As a consequence of the two arguments, we need discuss only the 
complementary read frame with aligned codons.

\subsection{Second rule: there exist double synonyms}
\label{sec: 2.2}

Most synonymous mutations of the forward strand produce a non-synonymous change of the 
complementary strand, but the genetic code does include a number of double synonyms, 
where the reverse complement of a synonymous mutation is also a synonym. 
For example codon AGG (Arg) can make a transversal mutation to CGG (Arg), 
while the reverse complement of AGG, which is CCU (Pro) transforms to CCG (Pro) under 
the same mutation. 

The full set of double synonyms in the standard genetic code are as follows:

\begin{itemize}

\item Two of the six synonyms of Ser are double synonyms, with reverse complements coding Arg. 
Conversely, two of the six synonyms of Arg are double synonyms, with reverse complement coding Ser.

\item Two more of the six synonyms of Arg are double synonyms, with reverse complement Pro. 
Conversely, two of the four synonyms of Pro are double synonyms coding for Arg.

\item Two of the six synonyms of Leu are double synonyms, with reverse complement Gln.
Conversely, both synonyms of Gln are double synonyms, with reverse complement coding Leu. 

\end{itemize}

Table \ref{tab: 1} lists the sets of single and double synonyms for those amino acids 
that can have double synonyms. (We exclude the two synonyms of Ser and the one synonym
of Leu for which the reverse complement is Stop, because these do not occur in ambigrammatic
genes.) 

\begin{table}
\begin{tabular}{|c|l|c|c|c|}
\hline
AA&Codon&\MW{Syns.: $S^{({\rm n})}+S^{({\rm v})}$}&\MW{Dbl syns.: $D^{({\rm n})}+D^{({\rm v})}$}&Comp. AA\\ 
\hline
Leu &UUG$\ast$&$1+0$&$1+0$&Gln\\ 
       &CUU&$1+1$&$0+0$&Lys\\ 
       &CUC&$1+1$&$0+0$&Glu\\ 
       &CUG$\ast$&$1+2$&$1+0$&Gln\\  
\hline
Pro &CCU$\ast$&$1+2$&$0+1$&Arg\\ 
       &CCC&$1+2$&$0+0$&Gly\\ 
       &CCA&$1+2$&$0+0$&Trp\\ 
       &CCG$\ast$&$1+2$&$0+1$&Arg\\  
\hline
Gln &CAA$\ast$&$1+0$&$1+0$&Leu\\ 
       &CAG$\ast$&$1+0$&$1+0$&Leu\\ 
\hline
Arg &CGU&$1+2$&$0+0$&Thr\\ 
       &CGC&$1+2$&$0+0$&Ala\\ 
       &CGA$\ast$&$1+3$&$0+1$&Ser\\ 
       &CGG$\ast$&$1+3$&$0+1$&Pro\\  
       &AGA$\ast$&$\MW{1}+1$&$0+1$&Ser\\ 
       &AGG$\ast$&$\MW{1}+1$&$0+1$&Pro\\ 
\hline
Ser &UCU$\ast$&$1+1$&$0+1$&Arg\\ 
       &UCC&$1+1$&$0+0$&Gly\\ 
       &UCG$\ast$&$1+2$&$0+1$&Arg\\ 
       &AGU&$1+0$&$0+0$&Thr\\  
       &AGC&$1+0$&$0+0$&Ala\\ 
\hline
\end{tabular}
\caption{For each amino acid (AA) that can have double-synonym mutations, we list 
all of the possible codons which do not code for Stop on the complementary strand, indicating 
their reverse complement (Comp. AA). 
The codons that have a double synonym are marked with an asterisk. 
For each of these codons, we list the number of mutations which are synonymous, 
and the number of double synonym mutations. \MW{In each case the numbers of single (double) mutations are written 
$S^{({\rm n})}+S^{({\rm v})}$ ($D^{({\rm n})}+D^{({\rm v})}$), where the superscript n denotes transitions, 
and superscript v transversions.}
\label{tab: 1}}
\end{table}

\subsection{Implications}
\label{sec: 2.3}

Our first rule shows that an ambigrammatic version of any gene can evolve, without 
making any changes to the amino acid sequence. This establishes how ambigrammatic sequences 
can arise, but it does not illuminate why they are favoured. 

Combined with data on polymorphism of the narnaviruses, the second ambigram rule can give an 
indication of the utility of ambigrammatic sequences. In studies on the (usual) non-ambigrammatic 
genomes, the ratio of synonymous to non-synonymous mutations is used as an indicator of 
whether the nucleotide sequence codes for a protein: non-synonymous mutations 
are likely to be deleterious if the sequence codes for a functional protein. 
We shall adapt this approach to our study of ambigrammatic narnavirus genes.
We assume that the forward direction is a coding sequence (usually for RdRp), 
and confine attention to those mutations which are synonymous in the forward 
direction. If the complementary strand codes for a functional protein, most of these 
synonymous mutations will inevitably result in changes of the complementary 
amino acid sequence. However, at \MW{many} loci the evolutionarily favoured amino acid 
will be one that allows double synonyms. In these cases, there can be 
non-deleterious mutations between a pair of codons that preserve the amino acid 
sequence of both the forward and the 
complementary strands.   

If the complementary strand codes for a functional protein, we expect studies of the 
polymorphism of the gene would show that these double-synonym loci will be 
mutational \lq hotspots', where mutations occur more frequently. In addition, the double-synonym pairs 
would be represented far more frequently than other mutations at these loci. These observations 
lead to two distinct tests for whether there is evolutionary pressure on the translated image of the 
complementary strand.

\section{Tests of whether the complementary strand is coding}
\label{sec: 3}

We have argued that doubly-synonymous mutations will give a signature of the 
reverse strand coding for a functional protein. If the reverse-direction code is functional, 
then the only assuredly non-deleterious mutations would be the double-synonym ones,
where one codon is transformed by a single-nucleotide substitution to another 
codon which preserves the amino acid coded in both the forward and 
the reverse directions. 

Assume that we have $M$ polymorphs of an ambigrammatic gene, fully sequenced and 
maximally aligned with each other, and that one strand, referred to as the 
\lq forward' strand, codes for a functional protein. 
We identify a \lq consensus' codon at each of the ${\cal N}$ \MW{loci, 
and then} enumerate the set of variant codons at each amino acid locus. 
If the consensus codon at a locus is one of the twelve 
double-synonym codons listed in table \ref{tab: 1}, we term this a \emph{double-synonym locus}.
The number of double-synonym loci is ${\cal N}_{\rm d}$.

There are two different approaches to testing whether 
double synonyms indicate the that the complementary strand is coding:

\subsection{Look for the existence of mutational \lq hotspots'}
\label{sec: 3.1} 

We can look for evidence that the double-synonym loci are more active than other loci. 

For each codon locus $k$, we can determine the number of elements of the variant set, $n(k)$, 
and also the fraction of codons $f(k)$ which differ from the consensus codon. 
We then determine the averages of these quantities, $\langle n(k)\rangle$ and $\langle f(k)\rangle$, 
for the double-synonym loci and for the other loci. If the ratios
%
\begin{equation}
\label{eq: 3.1}
R_n=\frac{\langle n(k)\rangle\vert_{\rm double\ syn.\ loci}}{\langle n(k)\rangle\vert_{\rm other\ loci}}
\ ,\ \ \ 
R_f=\frac{\langle f(k)\rangle\vert_{\rm double\ syn.\ loci}}{\langle f(k)\rangle\vert_{\rm other\ loci}}
\end{equation}
%
are large, this is evidence that the complementary strand is coding. 

The null hypothesis, indicating that the reverse open reading frame is 
non-coding, is that the ratios $R_n$ and $R_f$ are sufficiently close to 
unity that the difference may be explained by statistical fluctuations. In particular,
if $\delta R=|1-R|$, the deviation of $R$ from unity is significant if 
$\delta R \sqrt{{\cal N}}\gg 1$.

\subsection{Mutation frequencies test}
\label{sec: 3.2}

We can also look at codon frequencies for \MW{different} mutations at the double-synonym loci. 
If the complementary strand is coding, we expect to find that the frequency of mutations 
observed at double-synonym loci will heavily favour double-synonym codons over 
single-synonyms.  We consider the subset of double-synonym loci where mutations 
are observed (that is, where $n(k)>1$). For each of these ${\cal N}_{\rm a}$ \emph{mutationally 
active double-synonym loci}, we can determine two numbers: $n_{\rm s}(k)$ is 
the numbers of singly-synonymous variants at locus $k$, and $n_{\rm d}(k)$ is the number 
of these variants which are also doubly-synonymous. (Clearly $n(k)\ge n_{\rm s}(k)\ge n_{\rm d}(k)$). 
If $n_{\rm d}(k)=n_{\rm s}(k)$, that means that the mutations preserve the complementary-strand 
amino acid, which is an indication that the reverse strand is coding. If $\{k^\ast\}$ 
is the set of mutationally active double-synonym loci, we then
calculate
%
\begin{equation}
\label{eq: 3.2}
N_{\rm s}=\sum_{k\in\{k^\ast\}} n_{\rm s}(k)
\ ,\ \ \ 
N_{\rm d}=\sum_{k\in\{k^\ast\}} n_{\rm d}(k)
\ .
\end{equation}
%
If the complementary strand is coding, we expect 
%
\begin{equation}
\label{eq: 3.3}
R\equiv \frac{N_{\rm s}}{N_{\rm d}}
\end{equation}
%
to be close to unity.

However, there will also be beneficial 
or neutral mutations which do change the amino acids, so that not all mutations 
will be between sets of doubly-synonymous codons. We need to be able to 
quantify the extent to which finding other than double-synonym mutations is 
an indication that the reverse strand is non-coding. We must do this by comparison with 
a null hypothesis, in which the reverse strand is non-coding. 

\subsection{Null hypothesis for mutation frequencies}
\label{sec: 3.3}

We must estimate how large $R$ can be before the complementary-strand coding hypothesis must 
be rejected. To this end, we shall estimate $N_{\rm exp}$, the expected value of $N_{\rm s}$, 
based upon the null hypothesis that the complementary strand is non-coding. 
If $R_0$ is the value of the ratio $R$ that is derived from this null hypothesis, then: 
%
\begin{equation}
\label{eq: 3.4}
N_{\rm exp}=R_0 N_{\rm d}
\ .
\end{equation}
%
We describe the calculation of $R_0$ below. 

We may assume that the codons at different loci may be 
modelled as are being statistical independent so that the  
numbers $N_{\rm s}$ and $N_{\rm d}$ are subject to Poissonian 
counting statistics. We can estimate the significance of the difference 
between $N_{\rm exp}$ and $N_{\rm s}$ by determining
%
\begin{equation}
\label{eq: 3.5}
\sigma=\frac{N_{\rm exp}-N_{\rm s}}{\sqrt{N_{\rm s}}}
\ .
\end{equation}
%
A large, positive, value of $\sigma$ would indicate that the neutrality hypothesis 
can be rejected.

\MW{We assume that the $M$ polymorphs are sufficiently similar
that only a small fraction of loci have undergone mutations. 
We adopt the Kimura model \cite{Kim80}, which assumes that the mutation
rates for transitions (${\rm A}\leftrightarrow {\rm G}$ or ${\rm C}\leftrightarrow{\rm U}$) are 
different from those of transversions (other single-nucleotide mutations), and negligible for 
other types of mutation. The ratio of these rates is  
%
\begin{equation}
\label{eq: 3.6}
\alpha=\frac{{\cal R}_{\rm transition}}{{\cal R}_{\rm transversion}}
\ .
\end{equation}
%
If the numbers of single (double) synonyms of the consensus nucleotide at locus $k$ leading to 
transitions or transversions are respectively $S^{({\rm n})}_k$ and $S^{({\rm v})}_k$ 
($D^{({\rm n})}_k$, $D^{({\rm v})}_k$), then we estimate
%
\begin{equation}
\label{eq: 3.7}
R_0=
\frac{\sum_{k\in\{k^\ast\}}\alpha S_k^{({\rm n})}+S^{({\rm v})}_k}
{\sum_{k\in\{k^\ast\}}\alpha D^{({\rm n})}_k+D_k^{({\rm v})}}
\end{equation}
%
The numbers $S^{({\rm n})}_k$, $S^{({\rm v})}_k$, $D^{({\rm n})}_k$, $D^{({\rm v})}_k$ are given 
in table \ref{tab: 1} for all of the double-synonym codons.}

\subsection{Non-synonymous to synonymous ratio test}
\label{sec: 3.4}

One standard test of whether a sequence codes for a protein is to look at the 
ratio of non-synonymous to synonymous mutations. 
We expect this ratio to be small when a readable base sequence is a functional gene coding for 
a well-conserved protein. Let us consider how to apply this test to the complementary strand 
of an ambigrammatic gene, which we assume is well-conserved in the forward direction. 
We can also consider another null-hypothesis for this test, in order to indicate whether the 
complementary strand codes for a functional protein. 

Let us assume that the forward strand is perfectly conserved (so that only synonymous 
mutations are allowed), and the complementary strand is non-coding. We shall see that there are some 
non-synonymous mutations for the forward strand of the narnavirus RdRp gene, but nevertheless applying this 
test does give additional insight. Let $\tilde N_{\rm n}$ and $\tilde N_{\rm s}$ 
be the \emph{total} numbers of non-synonymous and synonymous changes on the \emph{complementary} strand, 
summed over all codon loci. We evaluate 
%
\begin{equation}
\label{eq: 3.8}
\tilde R=\frac{\tilde N_{\rm n}}{\tilde N_{\rm s}}
\end{equation}
%
and compare it with an estimate $\tilde R_0$ which is derived from the null hypothesis that the complementary 
strand is non-coding. The value of $\tilde R_0$ is obtained by noting that, because the mutations on the forward 
strand are assumed to be synonymous, the total rates for non-synonymous and synonymous transitions 
on the complementary strands are proportional to $\alpha S^{({\rm n})}_k +S^{({\rm v})}_k$ and 
$\alpha D^{({\rm n})}_k +D^{({\rm v})}_k$ respectively, so that
%
\begin{equation}
\label{eq: 3.9}
\tilde R_0=\frac{\sum_k\alpha S_k^{({\rm n})}+S^{({\rm v})}_k}
{\sum_k \alpha D^{({\rm n})}_k+D_k^{({\rm v})}}
\end{equation}
%
where in this case we sum over \emph{all} of the codon loci for the complementary strand. 
(In order to implement this test we need the values of the coefficients $S^{({\rm n,v})}_k$ and 
$D^{({\rm n,v})}_k$ for all of the codons (other than stop codons), not just those listed in table \ref{tab: 1}.)

\section{Polymorphism of an ambigrammatic narnavirus}
\label{sec: 4}

In a recent study  \cite{Bat+20} of an ambigrammatic narnavirus, it was reported that this 
narnavirus system has the following properties:

\begin{enumerate}

\item There is a viral RNA segment which codes the for the RdRp, and which has the 
property of being ambigrammatic, with forward and reverse codons aligned, \MW{over very nearly 
the entire length (the forward strand has two stop codons close to the 3' end, and complementary strand 
has just one stop codon, which is also very close to its $3'$ end}).  

\item Infection with this sequence is strongly associated with the presence of another RNA 
sequence, which was referred to in \cite{Bat+20} as the \lq Robin' sequence. 

\item The Robin sequence is also ambigrammatic, over its entire length (about 850~nt), with the codons of the 
rORF aligned \MW{(in this case the strand that we designate as the forward has one stop close to the 3'
end, and its complementary strand has two stop codons, which are very close to its $3'$ end)}.  
Neither forward nor reverse directions are homologous \MW{with} known sequences.
 
\end{enumerate}

In addition $43$ polymorphic variants of both the narnavirus gene which codes for 
the RdRp and the Robin gene were sequenced. 
We determined the optimal optimal alignment of these $43$ sequences, and 
identified the \lq consensus' \MW{nucleotide (the one with the largest number of counts) at each locus.
We also identified the set of variant nucleotides observed at each locus, and counted the numbers
of transition and transversion mutations. The requirement that there be a minimal number of stops 
was used to determine the reading frames for both the forward and complementary sequences, and 
the consensus nucleotides were used to determine a consensus codon.}

\subsection{Results on the RdRp gene}
\label{sec: 4.1}

In order to bring all of the sequences into alignment we had to insert a few 'dummy' 
codons. \MW{We inserted either one or three dummy codons at nucleotide locus $395$ into 
most of the sequences,  and another dummy codon at nucleotide locus $2924$ into $19$ sequences.
For each nucleotide locus, we determined a consensus nucleotide, and determined the set of variants 
that were seen at each site. We found a total of $619$ transitions and $395$ transversions across the $3166$
nucleotide loci, indicating a mutation rate ratio $\alpha=3.13$, and a rate of selected mutations equal to 
$0.0074$ mutations per nucleotide per polymorph.}

For each codon locus $k$, we determined the number of elements of the variant set, $n(k)$, 
and also the fraction of codons $f(k)$ which differ from the consensus codon.
We identified the set of double synonym loci, for which the consensus codon is one of the 
starred codons listed in table \ref{tab: 1}. 

We first tested for whether there are mutational hotspots. We determined average values 
of $n(k)$ and $f(k)$ for the double synonym sites and for the other sites. The results are listed in table 
\ref{tab: 3}. From these data we find do not suggest that the double-synonym sites are mutational \lq hotspots'.

\begin{table}[h]
\begin{minipage}{0.5\textwidth}
\centering
\begin{tabular}{|c|c|c|}
\hline
Sample&$\langle n(k)\rangle$&$\langle f(k)\rangle$\\ 
\hline
Double syns., RdRp  &$0.943$&$0.193$\\ 
Other codons, RdRp  &$0.898$&$0.187$\\
\hline
\end{tabular}
\end{minipage}
\begin{minipage}{0.4\textwidth}
\centering
\begin{tabular}{|c|c|c|c|c|}
\hline
Gene&${\cal N}$&${\cal N}_{\rm d}$&$R_n$&$R_f$\\ 
\hline
RdRp  &$1054$&$227$&$1.049$&$1.034$\\ 
\hline
\end{tabular}
\end{minipage}
\caption{\MW{Summary of results of \lq mutational hotspots' test. Left panel:} values of the average number 
of elements of the variant set, $\langle n(k)\rangle$ and of the average fraction of non-consensus 
codons, $\langle f(k)\rangle$, for double-synonym sites,  and for the other sites.
Right panel: ${\cal N}$ is the number of loci in the alignment, ${\cal N}_{\rm d}$ is the 
number of double-synonym loci, and $R_n$, $R_f$ are the ratios defined in equation (\ref{eq: 3.1}). 
The differences of these ratios from unity do not appear significant.
\label{tab: 3}}
\end{table}

We then tried the mutational frequency test. Our results are presented in table \ref{tab: 4}. 
We found that $N_{\rm s}\gg N_{\rm d}$, indicating that there is not strong 
selection pressure on the complementary sequence.
Table \ref{tab: 4} also includes results on the application of the null hypothesis.
The value of $R_0$ depends upon $\alpha$: in table \ref{tab: 4} we used \MW{$\alpha=3.13$, 
as derived from our observed nucleotide mutations.}  The results are consistent with the null hypothesis, 
that the complementary sequence is non-coding. 

\begin{table}
\centering
\begin{tabular}{|c|c|c|c|c|c|c|}
\hline
Sample&${\cal N}$&${\cal N}_{\rm a}$&$N_{\rm s}$&$N_{\rm d}$&$R_0$&$\sigma$\\ 
\hline
RdRp  &$939$&$136$&$315$&$99$&$3.08$&$-0.52$\\ 
\hline
\end{tabular}
\caption{Results for the mutational codon frequency test: ${\cal N}$ is the number of loci
in the alignment, ${\cal N}_{\rm a}$ is the number of mutationally active double-synonym loci, and 
$N_{\rm s}$, $N_{\rm d}$ are, respectively, the numbers of single and double synonym mutations. 
The values of $R_0$ are evaluated at \MW{$\alpha=3.13$}.
\label{tab: 4}}
\end{table}

We conclude that there is no significant evidence that there are mutational hotspots at potentially 
doubly-synonymous sites, and that the frequencies of double synonym codons are compatible 
with what would be expected by chance, based upon the null hypothesis that the complementary 
strand is non-coding. 

\subsection{Results on the Robin gene}
\label{sec: 4.2}

We also investigated polymorphism of the Robin sequences, obtained from the same $43$ 
samples as the narnavirus RdRp genes. In order to bring all of the sequences into alignment 
we had to insert a few 'dummy' codons. \MW{We inserted two dummy codons at 
nucleotide locus $408$ into $25$ of the sequences,  and another dummy codon at nucleotide 
locus $560$ into $37$ sequences. For each nucleotide locus, we determined a consensus nucleotide, 
and determined the set of variants that were seen at each site. 
We found a total of $218$ transitions and $156$ transversions across the $860$
nucleotide loci, indicating a mutation rate ratio $\alpha=2.79$, and a rate of selected mutations equal to 
$0.0101$ mutations per nucleotide per polymorph.}

The next step was to identify which is the coding direction of the Robin gene. 
One expects that this can be identified by determining which read direction has 
a high proportion of synonymous mutations. We identified the consensus sequence 
for Robin, and evaluated the number of variants $n(k)$ at each locus, and the number 
of synonymous and non-synonymous variants, $n_{\rm syn}(k)$ and $n_{\rm nsyn}(k)$. 
In table \ref{tab: 5} we list the total of the number of non-synonymous 
and synonymous variants
%
\begin{equation}
\label{eq: 4.1}
N_{\rm nsyn}=\sum_k n_{\rm nsyn}(k)
\ ,\ \ \
N_{\rm syn}=\sum_k n_{\rm syn}(k)
\end{equation}
%
for the two different reading directions, denoted by Robin A and Robin B, together with comparable data 
for the forward and reverse reads of the RdRp gene.

In table \ref{tab: 5} we also indicate the number of mutations which involve more than changes to 
more than one nucleotide, $N_{\rm mult}$.

\begin{table}
\begin{tabular}{|c|c|c|c|c|c|}
\hline
Strand&$N_{\rm syn}$&$N_{\rm nsyn}$&$N_{\rm mult}$&$R=N_{\rm nsyn}/N_{\rm syn}$&$R_0$\\ 
\hline
Robin-fwd&$112$&$225$&$92$&$2.01$&$2.36$\\ 
Robin-comp&$68$&$261$&$89$&$3.84$&$2.42$\\ 
RdRp-fwd&$621$&$318$&$140$&$0.512$&$-$\\ 
RdRp-comp&$138$&$801$&$140$&$5.80$&$11.2$\\ 
\hline
\end{tabular}
\caption{Numbers of synonymous and non-synonymous mutations, $N_{\rm s}$ and $N_{\rm n}$, 
for two different reading directions of both genes. 
\label{tab: 5}}
\end{table}

\MW{In the case of the Robin gene, we see that the number of synonymous mutations $112$
is slightly higher than the null hypothesis for the forward strand, which predicts $100$ mutations.
The difference is not statistical significant. For the complementary strand there were $68$ synonymous 
mutations, which is significantly less than the predicted number of $96$. These results indicate that, if the 
Robin gene codes for a protein, it has been under very little selective pressure. 

However, there is evidence that the Robin gene has been under similar selective pressure to 
the the RdRp gene. The RdRp and Robin sequences are obtained from the same set of samples. 
The RdRp gene is regarded as a stable object which evolves very slowly. If Robin 
were under weak selective pressure, it would be expected to be more tolerant of mutations, so that 
its rate of selected mutations would be much higher. However, the rate of mutation per nucleotide per 
polymorph are quite close, differing by approximately $30\%$. 

Another line of evidence that Robin is under strong selective pressure comes from the number 
of codons which have undergone mutations at more than one nucleotide. Despite the fact that 
the rate of selected mutations per nucleotide per polymorph is small, the fraction of codons with 
multiple mutations is significant, for both the RdRp and Robin genes: we find that the fraction of 
multi-nucleotide mutations is ? for Robin-fwd, ? for Robin-comp and ? for RdRp-fwd. These data 
suggest that the Robin sequence has been under even more selective pressure than the RdRp 
gene.

While these lines of evidence are not conclusive, they do indicate that the Robin gene has been 
under selective pressure, without conserving the aminoacid sequence. The inference is that 
the Robin gene is not translated into a functional protein.} Because these results indicate that 
the Robin gene is not translated in either direction, so there is no point 
in testing for whether double synonyms are mutational hotspots. 

\section{Discussion}
\label{sec: 6}
  
We have argued that the recent observations of polymorphism of an ambigrammatic 
narnavirus \cite{Bat+20} favour the hypothesis that the reverse read of an ambigrammatic sequence does 
not code for a functional proteins. Furthermore, the Robin gene, which is symbiotic with the RdRp 
narnavirus gene, does not appear to code for a functional protein in either direction.

Other, circumstantial, evidence favours the interpretation that the complementary strand is non-coding.
Ambigrammatic sequences have been observed in a variety of simple RNA virus genomes , 
but they are undoubtedly a rare phenomenon. 
If the rORF (reverse open reading frame) of both the RdRp and the partner fragment 
had evolved to code for a functional protein, each RNA sequence would code for two genes.
Given that ambigrammatic sequences are rare \cite{DeR+19}, finding a system where two had evolved 
independently would be highly improbable. Moreover, because the ambigrams are full length, 
each of the ambigrammatically coded sequences would code for two genes which have the same 
length as each other.
An observation of the simultaneous detection of two or more ambigrammatic genes would 
strongly favour models where there is an advantage in evolving an 
ambigrammatic sequence which is independent of whether the complementary strand open reading 
frames are translated into functional proteins.

The role of the RdRp coding fragment 
is already understood. This makes it plausible that the other fragment plays a role which facilitates the 
evolution of ambigrams. If the lack of stop codons is on the complementary strand is not required 
to allow protein synthesis, we can surmise that its role is to allow ribosomes to associate with the 
complementary strand. Having RNA strands able to be covered by ribosomes may provide some protection 
for the viral RNA against degradation by the defence mechanisms of the host cell. 

The \lq Robin' sequence identified in \cite{Bat+20} appears to play an important role in the infection, 
because of its strong association with the RdRp coding sequence. A plausible hypothesis is that 
it somehow prevents ribosomes from detaching from the $3'$ end of viral RNA, 
so that the viral RNA can become shrouded in a coating of ribosomes. 
The evidence from polymorphism studies is consistent with the Robin RNA being able to 
bind directly with the RdRp gene, rather than being translated into a protein.
Further studies of \lq ribosome profiling' of this system have that potential to reveal evidence 
of whether ribosomes accumulate on the viral RNA in this system.  
 
\bibliography{polyvirus}


\section*{Acknowledgements}

We thank Hanna Retallack and Joe DeRisi for discussions of their experimental studies of narnaviruses
and Amy Kistler and Gitys Dudas for assistance with narnaviral genomes.
MW thanks the Chan Zuckerberg Biohub for its hospitality. 

\section*{Author contributions statement}

MW produced a draft of the manuscript following discussions with the other 
authors about the recent discovery of a narnavirus system which has two 
ambigrammatic genes. All authors contributed to writing the manuscript, and 
reviewed the manuscript before submission. 

\section*{Additional information}

There are no competing interests.

\end{document}


 










 

